\documentclass[a4paper]{article}

\usepackage[T1]{fontenc}
\usepackage[swedish]{babel}

\usepackage{hyperref}
\hypersetup{%
    pdfauthor = {Martin Barksten, Victor Koronen},
    pdftitle = {Compiler Construction},
    pdfsubject = {DD2488},
    pdfkeywords = {compiler},
    pdfcreator = {LaTeX with hyperref package},
    pdfproducer = {pdflatex}
}

\title{Compiler Construction}
\author{%
    Martin Barksten <\href{mailto:barksten@kth.se}{barksten@kth.se}> \and
    Victor Koronen <\href{mailto:koronen@kth.se}{koronen@kth.se}>
}
\date{\today}

\begin{document}

\maketitle
\thispagestyle{empty}

\section{Introduction}

This document describes the design and implementation of a compiler for a subset
of the Java language.

The document begins with a section about the structure of the program, following
that is a section giving an overview of the code and finally comes a section
about how to build, run and test the compiler.

\section{Structure}

In this section is a description of how the program is structured. The sections
come in the order of steps when compiling, beginning with the lexing and parsing
steps and ending with the assembling. The compiler compiles to the JVM and will
therefore skip some steps at the end, taking a shortcut via Jasmin to the JVM.

\subsection{Lexer and Parser}

The lexing and parsing are done using JavaCC, a Compiler Compiler for Java.
JavaCC handles the tokenization and allows the programmer to write the LL
grammar in \texttt{.jj} files that JavaCC then ``compiles'' to Java files. In
order to be able to use the LL grammar the given grammar had to have all
left-recursion eliminated, as well as priority given to Exp operators.

Eliminating left-recursion can be done simply in JavaCC by using the ability to
match $0$ or more possible productions. The grammar

$$A \rightarrow A + B$$

can be converted to

$$A \rightarrow B (+ B)*$$

where star represents the ability to match $0$ or more times. Rewriting all
rules this way all left recursion is eliminated.

The other problem is to give priority to operators. This is done by simply
following a rule. Assume we once again have the following grammar

$$A \rightarrow B + B$$
$$A \rightarrow B - B$$

And we want $-$ to have higher priority than $+$, we simply rewrite it as

$$A \rightarrow B + B$$
$$B \rightarrow C - C$$

This will give $-$ higher precendence than $+$.

\subsection{Abstract Syntax Tree}

Constructing the Abstract Syntax Tree (AST), requires first creating the
necessary classes to represent it. For each different terminator a class is
created to represent it. The AST is constructed during the parsing step, using
JavaCC. When, for example, a Plus operator is found we will continue the parsing
to find the left and right expression and then create the an instance of the
Plus object. This object is then returned from the grammar rule.

\subsection{Type checking}

The type checking is done in two passes, first one pass is made binding all
methods and variables to their types, and then a second pass is made checking
that all types match each other. In order to perform these two steps we need to
recursively visit the entire AST, which we do by using the Visitor pattern.

The visitor pattern allows us\ldots

In order to save the types we use a symbol table to map the identifier to the
type. We use three different symbol tables - a program table, a class table and
a method table. Each table will save different information. The program table
maps all class identifiers to their respective class table. The class tables map
the method identifiers to their respective method tables, it also keeps track of
what fields the class has as well as an optional super class. Finally the method
table maps variable identifiers to their types The method table also keeps track
of which variables are locals and and which are parameters.

When binding identifiers to variables we then go through the tree and whenever
we encounter the declaration of a class, method, field or variable we then
declare it in it's matching table. Classes are added to the program table,
methods and fields to the corresponding class table and variables to the
corresponding method table.

The program table, allowing access to all other tables, is then passed to the
Visitor used for the variable checking. In the second pass we once again go
through the tree checking that all operations on variables is correct.
Additionally when calling a method we check that the object it is called on has
the method defined for it. This is once again done using a visitor, and going
through the tree recursively.

\subsection{Assembling}


\section{Code overview}

\begin{description}
\item[\texttt{mjc}] Package containing executable main classes and the parser
    code.
    \begin{description}
        \item[\texttt{mjc.JVMMain}] The main executable.
    \end{description}
\end{description}

\section{Building, running and testing}

\end{document}
